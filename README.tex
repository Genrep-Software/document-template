%%%%%%%%%%%%%%%%%%%%%%%%%%%%%%%%%%%%%%%%%%%%%%%%%%%%%%%%%%%%%%%%%%%%%%%%%%%%%%%%
% Genrep Software, LLC. LaTeX document template
% Created June 2020
%
% Please modify this document before use. Make sure to include the title and
% author metadata in the preamble if compiling from Markdown.
%%%%%%%%%%%%%%%%%%%%%%%%%%%%%%%%%%%%%%%%%%%%%%%%%%%%%%%%%%%%%%%%%%%%%%%%%%%%%%%%


\documentclass{article}


%%%%%%%%%%%%%%%%%%%%%%%%%%%%%%%%%%%%%%%%%%%%%%%%%%%%%%%%%%%%%%%%%%%%%%%%%%%%%%%%
% Imports and formatting
%%%%%%%%%%%%%%%%%%%%%%%%%%%%%%%%%%%%%%%%%%%%%%%%%%%%%%%%%%%%%%%%%%%%%%%%%%%%%%%%

% Set the page on letter paper with microtype improvements; allow PDF import
\usepackage{microtype}
\usepackage[letterpaper]{geometry}
\usepackage{multicol}
\usepackage{pdfpages}
\usepackage{upquote}
\usepackage{longtable}
\usepackage{booktabs}

% Math imports
\usepackage{amsmath}
\usepackage{amssymb}
\usepackage{amsthm}

% Roboto typeface
\usepackage[sfdefault]{roboto}

% Fix issues with fonts when highlighting code
\usepackage[T1]{fontenc}

% Page header and fix page numbers to use the correct font
\usepackage{fancyhdr}
\fancyhf{}
\fancyfoot[C]{\thepage}
\pagestyle{fancyplain}

% For strikethrough text
\usepackage[normalem]{ulem}

% No indent and skip lines
\setlength{\parindent}{0em}
\setlength{\parskip}{1em}

% Add spacing after footnote numbers
\usepackage[hang, bottom]{footmisc}
\setlength{\footnotemargin}{0.75em}

% Format section headers
\usepackage{titlesec}
\titleformat{\section}{\Large\sc}{\thesection\quad}{0em}{}
\titlespacing{\section}{0em}{1em}{0em}
\titleformat{\subsection}{\large\sc}{\thesubsection\quad}{0em}{}
\titlespacing{\subsection}{0em}{1em}{0em}
\titleformat{\subsubsection}{\normalsize\sc}{\thesubsubsection\quad}{0em}{}
\titlespacing{\subsubsection}{0em}{1em}{0em}

% List formatting
\usepackage{enumerate}
\usepackage{enumitem}
\setlist[itemize]{topsep=0pt}
% Copied from pandoc template via `pandoc.exe -D latex`
\providecommand{\tightlist}{\setlength{\parskip}{0pt}}

% Listings for code
% Copied from pandoc template via `pandoc.exe -D latex`
\usepackage{listings}
\newcommand{\passthrough}[1]{#1}
% Auto-generated by PANDOC
\usepackage{color}
\usepackage{fancyvrb}
\newcommand{\VerbBar}{|}
\newcommand{\VERB}{\Verb[commandchars=\\\{\}]}
\DefineVerbatimEnvironment{Highlighting}{Verbatim}{commandchars=\\\{\}}
% Add ',fontsize=\small' for more characters per line
\newenvironment{Shaded}{}{}
\newcommand{\AlertTok}[1]{\textcolor[rgb]{1.00,0.00,0.00}{\textbf{#1}}}
\newcommand{\AnnotationTok}[1]{\textcolor[rgb]{0.38,0.63,0.69}{\textbf{\textit{#1}}}}
\newcommand{\AttributeTok}[1]{\textcolor[rgb]{0.49,0.56,0.16}{#1}}
\newcommand{\BaseNTok}[1]{\textcolor[rgb]{0.25,0.63,0.44}{#1}}
\newcommand{\BuiltInTok}[1]{#1}
\newcommand{\CharTok}[1]{\textcolor[rgb]{0.25,0.44,0.63}{#1}}
\newcommand{\CommentTok}[1]{\textcolor[rgb]{0.38,0.63,0.69}{\textit{#1}}}
\newcommand{\CommentVarTok}[1]{\textcolor[rgb]{0.38,0.63,0.69}{\textbf{\textit{#1}}}}
\newcommand{\ConstantTok}[1]{\textcolor[rgb]{0.53,0.00,0.00}{#1}}
\newcommand{\ControlFlowTok}[1]{\textcolor[rgb]{0.00,0.44,0.13}{\textbf{#1}}}
\newcommand{\DataTypeTok}[1]{\textcolor[rgb]{0.56,0.13,0.00}{#1}}
\newcommand{\DecValTok}[1]{\textcolor[rgb]{0.25,0.63,0.44}{#1}}
\newcommand{\DocumentationTok}[1]{\textcolor[rgb]{0.73,0.13,0.13}{\textit{#1}}}
\newcommand{\ErrorTok}[1]{\textcolor[rgb]{1.00,0.00,0.00}{\textbf{#1}}}
\newcommand{\ExtensionTok}[1]{#1}
\newcommand{\FloatTok}[1]{\textcolor[rgb]{0.25,0.63,0.44}{#1}}
\newcommand{\FunctionTok}[1]{\textcolor[rgb]{0.02,0.16,0.49}{#1}}
\newcommand{\ImportTok}[1]{#1}
\newcommand{\InformationTok}[1]{\textcolor[rgb]{0.38,0.63,0.69}{\textbf{\textit{#1}}}}
\newcommand{\KeywordTok}[1]{\textcolor[rgb]{0.00,0.44,0.13}{\textbf{#1}}}
\newcommand{\NormalTok}[1]{#1}
\newcommand{\OperatorTok}[1]{\textcolor[rgb]{0.40,0.40,0.40}{#1}}
\newcommand{\OtherTok}[1]{\textcolor[rgb]{0.00,0.44,0.13}{#1}}
\newcommand{\PreprocessorTok}[1]{\textcolor[rgb]{0.74,0.48,0.00}{#1}}
\newcommand{\RegionMarkerTok}[1]{#1}
\newcommand{\SpecialCharTok}[1]{\textcolor[rgb]{0.25,0.44,0.63}{#1}}
\newcommand{\SpecialStringTok}[1]{\textcolor[rgb]{0.73,0.40,0.53}{#1}}
\newcommand{\StringTok}[1]{\textcolor[rgb]{0.25,0.44,0.63}{#1}}
\newcommand{\VariableTok}[1]{\textcolor[rgb]{0.10,0.09,0.49}{#1}}
\newcommand{\VerbatimStringTok}[1]{\textcolor[rgb]{0.25,0.44,0.63}{#1}}
\newcommand{\WarningTok}[1]{\textcolor[rgb]{0.38,0.63,0.69}{\textbf{\textit{#1}}}}

% Handle proper formatting for images and graphics


%%%%%%%%%%%%%%%%%%%%%%%%%%%%%%%%%%%%%%%%%%%%%%%%%%%%%%%%%%%%%%%%%%%%%%%%%%%%%%%%
% Custom commands -- redefined commands that should look or behave differently
%%%%%%%%%%%%%%%%%%%%%%%%%%%%%%%%%%%%%%%%%%%%%%%%%%%%%%%%%%%%%%%%%%%%%%%%%%%%%%%%

% Redefine the ampersand to use the italic version by default - as-per Elements
% of Typographic Style
% See: https://tex.stackexchange.com/a/47353/150811
\let\textand\&
\renewcommand{\&}{\textit{\textand}}

% Redefine the maketitle command to format properly, and not have a header on
% the page
\let\origtitle\maketitle
\renewcommand{\maketitle}{
    \setlength{\parskip}{0em}
    \origtitle
    \thispagestyle{empty}
    \setlength{\parskip}{1em}
}

% Redefine the table of contents command to format properly
\let\origtoc\tableofcontents
\renewcommand{\tableofcontents}{
    \setlength{\parskip}{0em}
    \origtoc
    \thispagestyle{empty}
    \setlength{\parskip}{1em}
}

% Make all graphics fit the page
% \let\origgraphic\includegraphics
% \renewcommand{\includegraphics}[1]{\origgraphic[width=\linewidth]{#1}}
\newcommand{\includeimage}[1]{\includegraphics[width=\linewidth]{#1}}

% Linked logos to use in the document title and header
\newcommand{\biglogo}{
  \href{https://genrep.software}{\includegraphics[width=0.4\linewidth]{logo}}
}
\newcommand{\smalllogo}{
  \href{https://genrep.software}{\includegraphics[width=0.25\linewidth]{logo}}
}


%%%%%%%%%%%%%%%%%%%%%%%%%%%%%%%%%%%%%%%%%%%%%%%%%%%%%%%%%%%%%%%%%%%%%%%%%%%%%%%%
% Document metadata
%%%%%%%%%%%%%%%%%%%%%%%%%%%%%%%%%%%%%%%%%%%%%%%%%%%%%%%%%%%%%%%%%%%%%%%%%%%%%%%%

\usepackage[pdftex,
    pdftitle={Document Typesetting Template Readme -- Draft},
    pdfauthor={Genrep Software, LLC.}]{hyperref}

\title{Document Typesetting Template Readme -- Draft}
\author{\biglogo}
\date{June 16, 2020}

% Display logo on the left part of the header and document info at right
\lhead{\smalllogo}
\rhead{Document Typesetting Template Readme -- Draft \\ June 16, 2020}



%%%%%%%%%%%%%%%%%%%%%%%%%%%%%%%%%%%%%%%%%%%%%%%%%%%%%%%%%%%%%%%%%%%%%%%%%%%%%%%%
% Document
%%%%%%%%%%%%%%%%%%%%%%%%%%%%%%%%%%%%%%%%%%%%%%%%%%%%%%%%%%%%%%%%%%%%%%%%%%%%%%%%

\begin{document}

\maketitle
\newpage
\tableofcontents
\newpage


% Auto-generated by PANDOC
\hypertarget{preamble}{%
\section{Preamble}\label{preamble}}

The text above is a ``document preamble.'' It looks pretty normal within
the \texttt{README.md} Markdown file, and it is invisible in the
\texttt{README.pdf} generated by Pandoc. However it is sometimes visible
as a table in the GitHub Markdown view for the repository. In Markdown,
the YAML-formatted preamble looks like:

\begin{Shaded}
\begin{Highlighting}[]
\CommentTok{{-}{-}{-}}
\AnnotationTok{title:}\CommentTok{ Document Typesetting Template Readme {-}{-} Draft}
\AnnotationTok{author:}\CommentTok{ Genrep Software, LLC.}
\AnnotationTok{date:}\CommentTok{ June 16, 2020}
\CommentTok{{-}{-}{-}}
\end{Highlighting}
\end{Shaded}

If reading the PDF, notice that the above has fancy syntax highlighting.
Pandoc does this for us! Also note that three variables are set:
\texttt{title}, \texttt{author}, and \texttt{date}. All of these must be
set for the document to compile properly. If you are converting a
non-Markdown document and cannot set the metadata values, run Pandoc (or
the compilation script) with the \texttt{-M} or
\texttt{-\/-metadata}\footnote{These two options are completely
  interchangeable. \texttt{-M} makes more sense for use in the terminal,
  whereas \texttt{-\/-metadata} makes much more sense for scripts where
  readability is a concern.} option as follows, for example:

\begin{Shaded}
\begin{Highlighting}[]
\ExtensionTok{pandoc}\NormalTok{ {-}M title:}\StringTok{"Fancy title"}\NormalTok{ {-}{-}metadata author:}\StringTok{"Genrep"}\NormalTok{ README.md {-}o README.tex}
\end{Highlighting}
\end{Shaded}

Note that this technique of manually including metadata is particularly
useful for compiling \texttt{.docx} Microsoft Word documents, which may
have been exported and downloaded from Google Docs. Look in the
\texttt{compile.sh} script for an exmample of how this is used. Read the
\protect\hyperlink{usage}{Usage} section to learn about how to properly
compile \texttt{.docx} documents using the compilation script.

Unfortunately, even though Markdown syntax is similar across platforms
that use it, there is no official standard. When writing Markdown
documents to version control and eventually typeset using this template
and script, refer to the
\href{https://pandoc.org/MANUAL.html\#pandocs-markdown}{Pandoc Markdown
Reference}.\footnote{\url{https://pandoc.org/MANUAL.html\#pandocs-markdown}}

\hypertarget{introduction}{%
\section{Introduction}\label{introduction}}

Genrep Software specializes in, among other things, report generation.
As a result, it is absolutely critical that the documents put out by the
company look clean and professional. Using an advanced typesetting
system is a good way to rapidly create professional-looking documents
with highly consistent formatting. Furthermore, scripting the process of
typesetting documents aligns with our mission of automating report
generation.

For typesetting, we primarily use \LaTeX{} (LaTeX). \LaTeX{} is an
advanced document typesetting system, originally created decades ago by
famed computer scientist Donald Knuth. Pandoc compiles Markdown to
\LaTeX{} for us, so that we don't have to learn and remember the fancy
\LaTeX{} commands. That being said, it is possible to inline \LaTeX{}
commands. For example, if reading in the PDF, it will not be obvious
that the weird-looking word \texttt{LaTeX} is created by typing
\texttt{\textbackslash{}LaTeX\{\}} into the Markdown. GitHub renders the
Markdown without converting it to \LaTeX{}, so it will look like raw
commands, rather than nice formatting there.

One benefit of using \LaTeX{} is that we can also inline math, which
will be beautifully typeset for us. The nice math below will not render
correctly on GitHub unfortunately. To see how nice the math looks, view
the \texttt{README.pdf} file in the repository.

\[ \mathrm{Borwein} (n) := \int_0^\infty \prod_{i = 0}^n \frac{\sin (x)}{x} dx \]

The code to render the math above is:

\begin{Shaded}
\begin{Highlighting}[]
\SpecialStringTok{$$ }\SpecialCharTok{\textbackslash{}mathrm}\SpecialStringTok{\{Borwein\} (n) := }\SpecialCharTok{\textbackslash{}int}\SpecialStringTok{\_0\^{}}\SpecialCharTok{\textbackslash{}infty}\SpecialStringTok{ }\SpecialCharTok{\textbackslash{}prod}\SpecialStringTok{\_\{i = 0\}\^{}n }\SpecialCharTok{\textbackslash{}frac}\SpecialStringTok{\{}\SpecialCharTok{\textbackslash{}sin}\SpecialStringTok{ (x)\}\{x\} dx $$}
\end{Highlighting}
\end{Shaded}

\hypertarget{quick-start}{%
\section{Quick Start}\label{quick-start}}

\hypertarget{setup}{%
\subsection{Setup}\label{setup}}

Make sure that you have Pandoc and \LaTeX{} installed locally. There are
other ways to compile \texttt{.tex} files, such as
\href{https://www.overleaf.com/}{Overleaf},\footnote{\url{https://www.overleaf.com/}}
but for now, it is easiest to have a local distribution running.
Installation instructions can be found at the following links
(respectively):

\begin{itemize}
\tightlist
\item
  \url{https://pandoc.org/installing.html}
\item
  \url{http://www.tug.org/texlive/acquire-netinstall.html}
\end{itemize}

Once these are set up, clone the repository for access to the
compilation script, logo image, and document template.

\begin{Shaded}
\begin{Highlighting}[]
\FunctionTok{git}\NormalTok{ clone https://github.com/Genrep{-}Software/document{-}template.git }\KeywordTok{\&\&}
\BuiltInTok{cd}\NormalTok{ document{-}template}
\end{Highlighting}
\end{Shaded}

To test your Pandoc/\LaTeX{} installation, try compiling the
\texttt{README.md} file from within the \texttt{document-template}
repository directory. Make sure the resulting PDF file is viewable.

\begin{Shaded}
\begin{Highlighting}[]
\ExtensionTok{./compile.sh}\NormalTok{ README.md}
\end{Highlighting}
\end{Shaded}

\hypertarget{usage}{%
\subsection{Usage}\label{usage}}

Using the template is just a matter of cloning the repository and
running the \texttt{compile.sh} script. If there is no \texttt{logo.png}
or \texttt{template.tex} in the working directory where the script is
called, it will automatically pull the latest versions from the GitHub
repository.

\hypertarget{markdown-files}{%
\subsubsection{Markdown Files}\label{markdown-files}}

Pandoc-flavored Markdown files can contain metadata (title, author, and
date) in \href{https://pandoc.org/MANUAL.html\#metadata-blocks}{YAML}
format.\footnote{\url{https://pandoc.org/MANUAL.html\#metadata-blocks}}
This data is used while compiling the document, and having a
\texttt{title} field, an \texttt{author} field, and a \texttt{date}
field is \emph{absolutely required}. If any of these fields is not
included, the document may be formatted strangely, and \LaTeX{} may give
cryptic errors. When debugging this script and the template, the first
thing to check is whether the test case has the proper metadata. This
script assumes that if a Markdown file is being typeset, it will have
data inside.

To compile a Markdown file with YAML metadata inside the file, use:

\begin{Shaded}
\begin{Highlighting}[]
\ExtensionTok{./compile.sh} \OperatorTok{\textless{}}\NormalTok{filename to convert (ending in .md)}\OperatorTok{\textgreater{}}
\end{Highlighting}
\end{Shaded}

\hypertarget{google-docs-links}{%
\subsubsection{Google Docs Links}\label{google-docs-links}}

For comiling Google Docs links, it is necessary to include the URL and a
title. There are also optional \texttt{author} and \texttt{date}
arguments, but ``Genrep Software, LLC.'' and the date of compilation
will be used if they are not provided. Make sure to surround the
command-line arguments with quotation marks.

Also ensure that any Google Docs link used has the correct sharing
settings. In particular, it must be set so that ``Anyone with the link
can view,'' that way this script can access the link to export the
document. If the sharing permissions are set incorrectly, there may be
cryptic errors due to \LaTeX{} trying to compile a mostly empty
document.

The general usage of the command for Google Docs links is:

\begin{Shaded}
\begin{Highlighting}[]
\ExtensionTok{./compile.sh} \OperatorTok{\textless{}}\NormalTok{Google Docs link}\OperatorTok{\textgreater{}} \StringTok{"\textless{}title\textgreater{}"} \StringTok{"[author]"} \StringTok{"[date]"}
\end{Highlighting}
\end{Shaded}

In particular, since the \texttt{author} and \texttt{date} arguments are
optional, we can do:

\begin{Shaded}
\begin{Highlighting}[]
\ExtensionTok{./compile.sh}\NormalTok{ \textbackslash{}}
  \StringTok{"https://docs.google.com/document/d/1w{-}AzMzVo05u5aeyMHdYXJeiOwdfGFMCQxZ2Ks7\_QWw8/edit"}\NormalTok{ \textbackslash{}}
  \StringTok{"Old To{-}do List"}
\end{Highlighting}
\end{Shaded}

This will generate a nicely-formatted, typeset file called
\texttt{Old\ To-do\ List.pdf}.

\hypertarget{miscellaneous-document-files}{%
\subsubsection{Miscellaneous Document
Files}\label{miscellaneous-document-files}}

To compile any other type of file, it is necessary to manually include
the title, author, and date metadata. This can be done by specifying
them as command-line arguments for the compilation script in that order.
It is very important that the arguments be surrounded by quotation marks
if they have spaces in them. For example:

\begin{Shaded}
\begin{Highlighting}[]
\ExtensionTok{./compile.sh}\NormalTok{ proposal.docx }\StringTok{"This is the title"} \StringTok{"Genrep Software, LLC"} \StringTok{"4/20/69"}
\end{Highlighting}
\end{Shaded}

The general usage of the command for non-Markdown files is:

\begin{Shaded}
\begin{Highlighting}[]
\ExtensionTok{./compile.sh} \OperatorTok{\textless{}}\NormalTok{input file}\OperatorTok{\textgreater{}} \StringTok{"\textless{}title\textgreater{}"} \StringTok{"\textless{}author\textgreater{}"} \StringTok{"\textless{}date\textgreater{}"}
\end{Highlighting}
\end{Shaded}

The compilation script will throw an error and fail if it is given a
non-Markdown file with too few arguments. If, in the generated PDF, the
displayed title is only a single word and/or the author or date is only
a single word, make sure to surround the command-line arguments with
quotation marks.

\hypertarget{modifying-the-template}{%
\section{Modifying the Template}\label{modifying-the-template}}

Modifications to the template may be required for aesthetic reasons, or
if there are compilation errors due to necessary packages not being
imported. If there are errors because of packages, consult the
\texttt{pandoc-default-template.tex} file included in this repository.
The file was generated by running

\begin{Shaded}
\begin{Highlighting}[]
\ExtensionTok{pandoc}\NormalTok{ {-}D latex }\OperatorTok{\textgreater{}}\NormalTok{ pandoc{-}default{-}template.tex}
\end{Highlighting}
\end{Shaded}

The file will include various packages based on internal Pandoc
variables that are set during document compilation. Do some digging in
there based on the compilation error and see if it is possible to
determine what the missing package is. Google is your friend here.

\hypertarget{further-reading}{%
\section{Further Reading}\label{further-reading}}

This document primarily contains information specific to this script and
template. To learn more about Pandoc-flavored Markdown, or to contribute
effectively to this project, click the links below.

\begin{itemize}
\tightlist
\item
  \url{https://pandoc.org/MANUAL.html}
\item
  \url{https://www.tug.org/begin.html}
\end{itemize}


\end{document}
