%%%%%%%%%%%%%%%%%%%%%%%%%%%%%%%%%%%%%%%%%%%%%%%%%%%%%%%%%%%%%%%%%%%%%%%%%%%%%%%%
% Genrep Software, LLC. LaTeX document template
% Created June 2020
%
% Please modify this document before use. Make sure to include the title and
% author metadata in the preamble if compiling from Markdown.
%%%%%%%%%%%%%%%%%%%%%%%%%%%%%%%%%%%%%%%%%%%%%%%%%%%%%%%%%%%%%%%%%%%%%%%%%%%%%%%%


\documentclass{article}


%%%%%%%%%%%%%%%%%%%%%%%%%%%%%%%%%%%%%%%%%%%%%%%%%%%%%%%%%%%%%%%%%%%%%%%%%%%%%%%%
% Imports and formatting
%%%%%%%%%%%%%%%%%%%%%%%%%%%%%%%%%%%%%%%%%%%%%%%%%%%%%%%%%%%%%%%%%%%%%%%%%%%%%%%%

% Set the page on letter paper with microtype improvements; allow PDF import
\usepackage{microtype}
\usepackage[letterpaper]{geometry}
\usepackage{multicol}
\usepackage{pdfpages}
\usepackage{upquote}
\usepackage{fancyhdr}
\usepackage{longtable}
\usepackage{booktabs}

% Roboto typeface
\usepackage[sfdefault]{roboto}

% Fix issues with fonts when highlighting code
\usepackage[T1]{fontenc}

% For strikethrough text
\usepackage[normalem]{ulem}

% No indent and skip lines
\setlength{\parindent}{0em}
\setlength{\parskip}{1em}

% Format section headers
\usepackage{titlesec}
\titleformat{\section}{\Large\sc}{\thesection\quad}{0em}{}
\titlespacing{\section}{0em}{1em}{0em}
\titleformat{\subsection}{\large\sc}{}{0em}{}
\titlespacing{\subsection}{0em}{1em}{0em}

% List formatting
\usepackage{enumerate}
\usepackage{enumitem}
\setlist[itemize]{topsep=0pt}
% Copied from pandoc template via `pandoc.exe -D latex`
\providecommand{\tightlist}{\setlength{\parskip}{0pt}}

% Listings for code
% Copied from pandoc template via `pandoc.exe -D latex`
\usepackage{listings}
\newcommand{\passthrough}[1]{#1}
% Auto-generated by PANDOC
\usepackage{color}
\usepackage{fancyvrb}
\newcommand{\VerbBar}{|}
\newcommand{\VERB}{\Verb[commandchars=\\\{\}]}
\DefineVerbatimEnvironment{Highlighting}{Verbatim}{commandchars=\\\{\}}
% Add ',fontsize=\small' for more characters per line
\newenvironment{Shaded}{}{}
\newcommand{\AlertTok}[1]{\textcolor[rgb]{1.00,0.00,0.00}{\textbf{#1}}}
\newcommand{\AnnotationTok}[1]{\textcolor[rgb]{0.38,0.63,0.69}{\textbf{\textit{#1}}}}
\newcommand{\AttributeTok}[1]{\textcolor[rgb]{0.49,0.56,0.16}{#1}}
\newcommand{\BaseNTok}[1]{\textcolor[rgb]{0.25,0.63,0.44}{#1}}
\newcommand{\BuiltInTok}[1]{#1}
\newcommand{\CharTok}[1]{\textcolor[rgb]{0.25,0.44,0.63}{#1}}
\newcommand{\CommentTok}[1]{\textcolor[rgb]{0.38,0.63,0.69}{\textit{#1}}}
\newcommand{\CommentVarTok}[1]{\textcolor[rgb]{0.38,0.63,0.69}{\textbf{\textit{#1}}}}
\newcommand{\ConstantTok}[1]{\textcolor[rgb]{0.53,0.00,0.00}{#1}}
\newcommand{\ControlFlowTok}[1]{\textcolor[rgb]{0.00,0.44,0.13}{\textbf{#1}}}
\newcommand{\DataTypeTok}[1]{\textcolor[rgb]{0.56,0.13,0.00}{#1}}
\newcommand{\DecValTok}[1]{\textcolor[rgb]{0.25,0.63,0.44}{#1}}
\newcommand{\DocumentationTok}[1]{\textcolor[rgb]{0.73,0.13,0.13}{\textit{#1}}}
\newcommand{\ErrorTok}[1]{\textcolor[rgb]{1.00,0.00,0.00}{\textbf{#1}}}
\newcommand{\ExtensionTok}[1]{#1}
\newcommand{\FloatTok}[1]{\textcolor[rgb]{0.25,0.63,0.44}{#1}}
\newcommand{\FunctionTok}[1]{\textcolor[rgb]{0.02,0.16,0.49}{#1}}
\newcommand{\ImportTok}[1]{#1}
\newcommand{\InformationTok}[1]{\textcolor[rgb]{0.38,0.63,0.69}{\textbf{\textit{#1}}}}
\newcommand{\KeywordTok}[1]{\textcolor[rgb]{0.00,0.44,0.13}{\textbf{#1}}}
\newcommand{\NormalTok}[1]{#1}
\newcommand{\OperatorTok}[1]{\textcolor[rgb]{0.40,0.40,0.40}{#1}}
\newcommand{\OtherTok}[1]{\textcolor[rgb]{0.00,0.44,0.13}{#1}}
\newcommand{\PreprocessorTok}[1]{\textcolor[rgb]{0.74,0.48,0.00}{#1}}
\newcommand{\RegionMarkerTok}[1]{#1}
\newcommand{\SpecialCharTok}[1]{\textcolor[rgb]{0.25,0.44,0.63}{#1}}
\newcommand{\SpecialStringTok}[1]{\textcolor[rgb]{0.73,0.40,0.53}{#1}}
\newcommand{\StringTok}[1]{\textcolor[rgb]{0.25,0.44,0.63}{#1}}
\newcommand{\VariableTok}[1]{\textcolor[rgb]{0.10,0.09,0.49}{#1}}
\newcommand{\VerbatimStringTok}[1]{\textcolor[rgb]{0.25,0.44,0.63}{#1}}
\newcommand{\WarningTok}[1]{\textcolor[rgb]{0.38,0.63,0.69}{\textbf{\textit{#1}}}}

% Handle proper formatting for images and graphics


%%%%%%%%%%%%%%%%%%%%%%%%%%%%%%%%%%%%%%%%%%%%%%%%%%%%%%%%%%%%%%%%%%%%%%%%%%%%%%%%
% Custom commands -- redefined commands that should look or behave differently
%%%%%%%%%%%%%%%%%%%%%%%%%%%%%%%%%%%%%%%%%%%%%%%%%%%%%%%%%%%%%%%%%%%%%%%%%%%%%%%%

% Redefine the ampersand to use the italic version by default - as-per Elements
% of Typographic Style
% See: https://tex.stackexchange.com/a/47353/150811
\let\textand\&
\renewcommand{\&}{\textit{\textand}}

% Redefine the maketitle command to format properly, and not have a header on
% the page
\let\origtitle\maketitle
\renewcommand{\maketitle}{
    \setlength{\parskip}{0em}
    \origtitle
    \thispagestyle{empty}
    \setlength{\parskip}{1em}
}

% Redefine the table of contents command to format properly
\let\origtoc\tableofcontents
\renewcommand{\tableofcontents}{
    \setlength{\parskip}{0em}
    \origtoc
    \thispagestyle{empty}
    \setlength{\parskip}{1em}
}

% Make all graphics fit the page
% \let\origgraphic\includegraphics
% \renewcommand{\includegraphics}[1]{\origgraphic[width=\linewidth]{#1}}
\newcommand{\includeimage}[1]{\includegraphics[width=\linewidth]{#1}}

% Linked logos to use in the document title and header
\newcommand{\biglogo}{
  \href{https://genrep.software}{\includegraphics[width=0.4\linewidth]{logo}}
}
\newcommand{\smalllogo}{
  \href{https://genrep.software}{\includegraphics[width=0.25\linewidth]{logo}}
}


%%%%%%%%%%%%%%%%%%%%%%%%%%%%%%%%%%%%%%%%%%%%%%%%%%%%%%%%%%%%%%%%%%%%%%%%%%%%%%%%
% Document metadata
%%%%%%%%%%%%%%%%%%%%%%%%%%%%%%%%%%%%%%%%%%%%%%%%%%%%%%%%%%%%%%%%%%%%%%%%%%%%%%%%

\usepackage[pdftex,
    pdftitle={Document Typesetting Template Readme -- Draft},
    pdfauthor={Genrep Software, LLC.}]{hyperref}

\title{Document Typesetting Template Readme -- Draft}
\author{Genrep Software, LLC.}
\date{June 16, 2020}

% Display logo on the left part of the header and document info at right
\lhead{\smalllogo}
\rhead{Document Typesetting Template Readme -- Draft \\ June 16, 2020}
\pagestyle{fancyplain}



%%%%%%%%%%%%%%%%%%%%%%%%%%%%%%%%%%%%%%%%%%%%%%%%%%%%%%%%%%%%%%%%%%%%%%%%%%%%%%%%
% Document
%%%%%%%%%%%%%%%%%%%%%%%%%%%%%%%%%%%%%%%%%%%%%%%%%%%%%%%%%%%%%%%%%%%%%%%%%%%%%%%%

\begin{document}

\maketitle
\newpage
\tableofcontents
\newpage


% Auto-generated by PANDOC
\hypertarget{preamble}{%
\section{Preamble}\label{preamble}}

The text above is a ``document preamble.'' It looks pretty normal within
this actual Markdown file, and it is invisible in the output PDF
generated by Pandoc. However it is sometimes visible in the GitHub
README view for the repository. If the text above is not there, it looks
like this:

\begin{Shaded}
\begin{Highlighting}[]
\CommentTok{{-}{-}{-}}
\AnnotationTok{title:}\CommentTok{ Document Typesetting Template Readme {-}{-} Draft}
\AnnotationTok{author:}\CommentTok{ Genrep Software, LLC.}
\AnnotationTok{date:}\CommentTok{ June 16, 2020}
\CommentTok{...}
\end{Highlighting}
\end{Shaded}

Notice that the above has fancy syntax highlighting in the PDF, but may
have some text that says \texttt{\{\ .markdown\ \}} on GitHub.
Unfortunately, even though Markdown is similar across platforms that use
it, there is no official standard. When writing Markdown documents to
version control and eventually typeset using this template and script,
refer to the
\href{https://pandoc.org/MANUAL.html\#pandocs-markdown}{Pandoc Markdown
Reference}.\footnote{\url{https://pandoc.org/MANUAL.html\#pandocs-markdown}}

\hypertarget{introduction}{%
\section{Introduction}\label{introduction}}

\LaTeX{} is an advanced document typesetting system, originally created
decades ago by famed computer scientist Donald Knuth. Pandoc compiles
Markdown to \LaTeX{} for us, so that we don't have to learn and remember
the fancy \LaTeX{} commands. That being said, it is possible to inline
\LaTeX{} commands. For example, if reading in the PDF, it will not be
obvious that the weird-looking word \texttt{LaTeX} is created by typing
\texttt{\textbackslash{}LaTeX\{\}} into the Markdown. GitHub renders the
Markdown without converting it to \LaTeX{}, so it will look like raw
commands, rather than nice formatting there.

One benefit of using \LaTeX{} is that we can also inline math, which
will be beautifully typeset for us. The nice math below will not render
correctly on GitHub unfortunately.

\[ \mathrm{Borwein} (n) := \int_0^\infty \prod_{i = 0}^n \frac{\sin (x)}{x} dx \]

The code to render the math above is:

\begin{Shaded}
\begin{Highlighting}[]
\SpecialStringTok{$$ }\SpecialCharTok{\textbackslash{}mathrm}\SpecialStringTok{\{Borwein\} (n) := }\SpecialCharTok{\textbackslash{}int}\SpecialStringTok{\_0\^{}}\SpecialCharTok{\textbackslash{}infty}\SpecialStringTok{ }\SpecialCharTok{\textbackslash{}prod}\SpecialStringTok{\_\{i = 0\}\^{}n }\SpecialCharTok{\textbackslash{}frac}\SpecialStringTok{\{}\SpecialCharTok{\textbackslash{}sin}\SpecialStringTok{ (x)\}\{x\} dx $$}
\end{Highlighting}
\end{Shaded}



\end{document}
